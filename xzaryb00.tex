% Created 2019-12-06 Fri 00:50
% Intended LaTeX compiler: pdflatex
\documentclass[11pt,titlepage]{article}
\usepackage[utf8x]{inputenc}
\usepackage[T1]{fontenc}
\usepackage{graphicx}
\usepackage{grffile}
\usepackage{longtable}
\usepackage{wrapfig}
\usepackage{rotating}
\usepackage[normalem]{ulem}
\usepackage{amsmath}
\usepackage{textcomp}
\usepackage{amssymb}
\usepackage{capt-of}
\usepackage{hyperref}
\usepackage[a4paper,total={6.5in, 9in}]{geometry}
\usepackage{libertine}
\usepackage{minted}
\usepackage{float}
\usepackage[caption = false]{subfig}
\setminted{fontsize=\footnotesize}
\author{Jakub Zárybnický, xzaryb00}
\date{6. 12. 2019}
\title{Projekt z MSP\\\medskip
\large Logo VUT! VYSOKÉ UČENÍ TECHNICKÉ V BRNĚ Fakulta informačních technologií Čísla zadání: 21, 6 Cvičení - skupina: pátek, 9.00}
\hypersetup{
 pdfauthor={Jakub Zárybnický, xzaryb00},
 pdftitle={Projekt z MSP},
 pdfkeywords={},
 pdfsubject={},
 pdfcreator={Emacs 26.1 (Org mode 9.1.9)}, 
 pdflang={Czech}}
\begin{document}

\maketitle
(setq python-shell-interpreter "/nix/store/gpnm7i19lpj8p43mjrdw03d0hjalmskl-python3-3.7.5/bin/python")

\section{Zadání projektu z předmětu MSP}
\label{sec:org6c1c812}
\textbf{Každý student obdrží na cvičení konkrétní data (čísla ze seznamu), pro které vypracuje projekt.}
K vypracování můžete použít libovolné statistické programy.

\begin{enumerate}
\item Při kontrole výrobků byla sledována odchylka X [mm] jejich rozměru od
požadované velikosti. Naměřené hodnoty tvoří statistický soubor v listu
Data př. 1.

\begin{enumerate}
\item Proveďte roztřídění statistického souboru, vytvořte tabulku četností a
nakreslete histogramy pro relativní četnosti a relativní kumulativní
četnosti.
\item Vypočtěte aritmetický průměr, medián, modus, rozptyl a směrodatnou
odchylku.
\item Vypočtěte bodové odhady střední hodnoty, rozptylu a směrodatné odchylky.
\item Testujte předpoklad o výběru z normálního rozdělení Pearsonovým
(chí-kvadrát) testem na hladině významnosti 0,05.
\item Za předpokladu (bez ohledu na výsledek části d)), že statistický soubor
byl získán náhodným výběrem z normálního rozdělení, určete intervalové
odhady střední hodnoty, rozptylu a směrodatné odchylky se spolehlivostí
0,95 a 0,99.
\item Testujte hypotézu optimálního seřízení stroje, tj. že střední hodnota
odchylky je nulová, proti dvoustranné alternativní hypotéze, že střední
hodnota odchylky je různá od nuly, a to na hladině významnosti 0,05.
\item Ověřte statistickým testem na hladině významnosti 0,05, zda seřízení
stroje ovlivnilo kvalitu výroby, víte-li, že výše uvedený statistický
soubor 50-ti hodnot vznikl spojením dvou dílčích statistických souborů
tak, že po naměření prvních 20-ti hodnot bylo provedeno nové seřízení
stroje a pak bylo naměřeno zbývajících 30 hodnot.
\end{enumerate}

Návod: Oba soubory zpracujte neroztříděné. Testujte nejprve rovnost rozptylů
odchylek před a po seřízení stroje. Podle výsledku pak zvolte vhodný postup
pro testování rovnosti středních hodnot odchylek před a po seřízení stroje.

\item Měřením dvojice (Výška[cm], Váha[kg]) u vybraných studentů z FIT byl získán
dvourozměrný statistický soubor zapsaný po dvojicích v řádcích v listu
Data př. 2.

\begin{figure}
\subfloat[caption]{\begin{tabular}{ccc}
\hline
a&b&c\\
\hline
\end{tabular}}\\
\subfloat[caption]{\begin{tabular}{ccc}
\hline
d&e&f\\
\hline
\end{tabular}}
\end{figure}

\begin{enumerate}
\item Vypočtěte bodový odhad koeficientu korelace.
\item Na hladině významnosti 0,05 testujte hypotézu, že náhodné veličiny Výška a
Váha jsou lineárně nezávislé.
\item \textbf{Regresní analýza} - data proložte přímkou: \(Vaha = \beta_0 + \beta_1 \times Vyska\)
\begin{enumerate}
\item Bodově odhadněte \(\beta_0\), \(\beta_1\) a rozptyl \(s_2\).
\item Na hladině významnosti 0,05 otestujte hypotézy:
\[H : \beta_0 = -100, H_A : \beta_0 \neq -100,\]
\[H : \beta_1 = 1, H_A : \beta_1 \neq 1,\]
\item Vytvořte graf bodů spolu s regresní přímkou a pásem spolehlivosti pro
individuální hodnotu výšky.
\end{enumerate}
\end{enumerate}
\end{enumerate}

\textbf{Termín pro odevzdání práce je 11 týden výuky zimního semestru ve cvičení.}

\begin{center}
\label{tab:orgc05cd70}
\begin{tabular}{lrr}
a & 1 & 2\\
b & 2 & 3\\
c & 3 & 4\\
\end{tabular}
\end{center}


\begin{minted}[]{python}
(x, data)
\end{minted}

\begin{verbatim}
# Out[17]:
: (2, [['a', 1, 2], ['b', 2, 3], ['c', 3, 4]])
\end{verbatim}

\begin{minted}[]{python}
_ = plt.hist(np.random.randn(20000), bins=200)
\end{minted}

\begin{center}
\includegraphics[width=.9\linewidth]{./obipy-resources/9LEZBV.png}
\end{center}

\section{Příklad 1}
\label{sec:org6e26c23}
\textbf{Při kontrole výrobků byla sledována odchylka X [mm] jejich rozměru od požadované velikosti.
Naměřené hodnoty tvoří statistický soubor v listu Data př. 1.}

X [mm]
1,83
0,98
-0,09
-0,23
2,56
0,31
1,06
0,01
0,75
2,26
-0,59
0,90
1,66
0,36
2,19
1,24
-0,58
0,79
0,02
0,31
1,61
0,75
2,46
0,86
0,63
-0,98
-0,75
2,67
1,79
1,84
0,49
1,68
0,39
-0,84
1,49
1,50
1,70
3,40
1,40
0,27
0,48
0,27
1,41
0,55
1,20
-0,68
1,59
0,80
1,21
-1,31

\section{Příklad 2}
\label{sec:orgf13985c}
\textbf{Měřením dvojice (Výška[cm], Váha[kg]) u vybraných studentů z FIT byl získán dvourozměrný
statistický soubor zapsaný po dvojicích v řádcích v listu Data př. 2.}

Výška [cm]	Váha [kg]
150	50
177	73
154	53
152	44
169	69
200	94
196	99
181	74
152	50
172	74
152	58
150	46
178	78
154	57
190	90
195	98
182	80
184	88
156	42
154	66
\end{document}
%%% Local Variables:
%%% mode: latex
%%% TeX-master: t
%%% End:
